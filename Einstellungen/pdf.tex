% Ermöglicht die Erweiterung von pdf-Dateien mit Links und Ähnlichem.
\usepackage{hyperref}
% Wenn Linkstellen im Text farbig sein sollen oder gar Formeln gesetzt werden, \texorpdfstring nutzen
% Mit \hypersetup werden die Schalter dieses Pakets weiter modifiziert

% Dokumenttitel setzen
\hypersetup{pdftitle={\Titel}}

% Autor setzen
\hypersetup{pdfauthor={\Autor}}

% Thema setzen
\hypersetup{pdfsubject={\Untertitel}}

% Einstellung der Linkfarben (Drucken oder reine PDF-Ansicht)
\ifcase\pdftype 
\or % Für Drucken
\hypersetup{
    colorlinks=true,
    citecolor=black,
    filecolor=black,
    linkcolor=black,
    menucolor=black,
    urlcolor=black
}
\or % Für reine PDF-Verwendung
\hypersetup{
   colorlinks=true,
   linkcolor=blue,
	 anchorcolor=black,
   citecolor=blue,
   filecolor=magenta,
   menucolor=blue,
   urlcolor=blue
}
\fi

% Einstellung, was das Inhaltsverzeichnis als Link verwenden soll
\ifcase\pdftype 
\or % Für Drucken
\hypersetup{linktoc=all}
\or % Für reine PDF-Verwendung
\hypersetup{linktoc=page}
\fi

% Einstellen der Bookmarks
\hypersetup{
bookmarksopen=true,
bookmarksnumbered=true
}

% Setzt Links korrekt für Floats (Fixt, das Labels nicht mehr am Ende stehen müssen)
\usepackage[all]{hypcap}

% Schönere URL-Formatierung (Für BibTeX und Text) von http://www.kronto.org/thesis/tips/url-formatting.html
\makeatletter
\def\url@leostyle{\@ifundefined{selectfont}{\def\UrlFont{\sf}}{\def\UrlFont{\small\ttfamily}}}
\makeatother
\urlstyle{leo}
% Manuelles Hinzufügen einer URL mit \url{www.narf.com}